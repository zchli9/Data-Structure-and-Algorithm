\documentclass[UTF8]{ctexart}
\usepackage{geometry, CJKutf8}
\geometry{margin=1.5cm, vmargin={0pt,1cm}}
\setlength{\topmargin}{-1cm}
\setlength{\paperheight}{29.7cm}
\setlength{\textheight}{25.3cm}

% useful packages.
\usepackage{amsfonts}
\usepackage{amsmath}
\usepackage{amssymb}
\usepackage{amsthm}
\usepackage{enumerate}
\usepackage{graphicx}
\usepackage{multicol}
\usepackage{fancyhdr}
\usepackage{layout}
\usepackage{listings}
\usepackage{float, caption}
\usepackage{listings}
\usepackage[usenames,dvipsnames]{xcolor}
\lstset{
    basicstyle=\ttfamily, basewidth=0.5em
}
\definecolor{mygreen}{rgb}{0,0.6,0}
\definecolor{mygray}{rgb}{0.5,0.5,0.5}
\definecolor{mymauve}{rgb}{0.58,0,0.82}
\lstset{ %
backgroundcolor=\color{white},   % choose the background color
basicstyle=\footnotesize\ttfamily,        % size of fonts used for the code
columns=fullflexible,
breaklines=true,                 % automatic line breaking only at whitespace
captionpos=b,                    % sets the caption-position to bottom
tabsize=4,
commentstyle=\color{mygreen},    % comment style
escapeinside={\%*}{*)},          % if you want to add LaTeX within your code
keywordstyle=\color{blue},       % keyword style
stringstyle=\color{mymauve}\ttfamily,     % string literal style
frame=single,
rulesepcolor=\color{red!20!green!20!blue!20},
% identifierstyle=\color{red},
language=c++,
}

% some common command
\newcommand{\dif}{\mathrm{d}}
\newcommand{\avg}[1]{\left\langle #1 \right\rangle}
\newcommand{\difFrac}[2]{\frac{\dif #1}{\dif #2}}
\newcommand{\pdfFrac}[2]{\frac{\partial #1}{\partial #2}}
\newcommand{\OFL}{\mathrm{OFL}}
\newcommand{\UFL}{\mathrm{UFL}}
\newcommand{\fl}{\mathrm{fl}}
\newcommand{\op}{\odot}
\newcommand{\Eabs}{E_{\mathrm{abs}}}
\newcommand{\Erel}{E_{\mathrm{rel}}}

\begin{document}

\pagestyle{fancy}
\fancyhead{}
\lhead{陈力豪, 3220103614}
\chead{数据结构与算法第六次作业}
\rhead{Nov.7th, 2024}

\section{$remove$函数设计思路}
添加几个函数:\par
int getBalance(BinaryNode *t) const; 计算左右子树高度差,判断当前树是否平衡,返回左子树和右子树的高度差\par
BinaryNode* rotateLeft(BinaryNode *t); 返回左旋后的树\par
BinaryNode* rotateRight(BinaryNode *t); 返回右旋后的树\par
BinaryNode* balance(BinaryNode *t); 综合运用上面的函数,返回平衡树\par
在romove中,remove某个元素后,最后一行加上t = balance(t),保证平衡即可。\par
\section{测试结果}
程序运行测试结果如下:
\begin{lstlisting}
    ......
    (1687148133,-686249595)
    (2027365225,1661557892)
    Empty tree
    ------------------------------
    Empty tree
    
    real	0m2.546s
    user	0m2.424s
    sys	0m0.046s    
\end{lstlisting}
用时少于3秒,且没有其他错误。\par
我用 valgrind 进行测试,发现没有发生内存泄露。
\begin{lstlisting}
    ==3570== HEAP SUMMARY:
    ==3570==     in use at exit: 0 bytes in 0 blocks
    ==3570==   total heap usage: 300,110 allocs, 300,110 frees, 9,679,992 bytes allocated
    ==3570== 
    ==3570== All heap blocks were freed -- no leaks are possible
    ==3570== 
    ==3570== For lists of detected and suppressed errors, rerun with: -s
    ==3570== ERROR SUMMARY: 0 errors from 0 contexts (suppressed: 0 from 0)
\end{lstlisting}
\end{document}

%%% Local Variables: 
%%% mode: latex
%%% TeX-master: t
%%% End: 
