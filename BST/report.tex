\documentclass[UTF8]{ctexart}
\usepackage{geometry, CJKutf8}
\geometry{margin=1.5cm, vmargin={0pt,1cm}}
\setlength{\topmargin}{-1cm}
\setlength{\paperheight}{29.7cm}
\setlength{\textheight}{25.3cm}

% useful packages.
\usepackage{amsfonts}
\usepackage{amsmath}
\usepackage{amssymb}
\usepackage{amsthm}
\usepackage{enumerate}
\usepackage{graphicx}
\usepackage{multicol}
\usepackage{fancyhdr}
\usepackage{layout}
\usepackage{listings}
\usepackage{float, caption}
\usepackage{listings}
\usepackage[usenames,dvipsnames]{xcolor}
\lstset{
    basicstyle=\ttfamily, basewidth=0.5em
}
\definecolor{mygreen}{rgb}{0,0.6,0}
\definecolor{mygray}{rgb}{0.5,0.5,0.5}
\definecolor{mymauve}{rgb}{0.58,0,0.82}
\lstset{ %
backgroundcolor=\color{white},   % choose the background color
basicstyle=\footnotesize\ttfamily,        % size of fonts used for the code
columns=fullflexible,
breaklines=true,                 % automatic line breaking only at whitespace
captionpos=b,                    % sets the caption-position to bottom
tabsize=4,
commentstyle=\color{mygreen},    % comment style
escapeinside={\%*}{*)},          % if you want to add LaTeX within your code
keywordstyle=\color{blue},       % keyword style
stringstyle=\color{mymauve}\ttfamily,     % string literal style
frame=single,
rulesepcolor=\color{red!20!green!20!blue!20},
% identifierstyle=\color{red},
language=c++,
}

% some common command
\newcommand{\dif}{\mathrm{d}}
\newcommand{\avg}[1]{\left\langle #1 \right\rangle}
\newcommand{\difFrac}[2]{\frac{\dif #1}{\dif #2}}
\newcommand{\pdfFrac}[2]{\frac{\partial #1}{\partial #2}}
\newcommand{\OFL}{\mathrm{OFL}}
\newcommand{\UFL}{\mathrm{UFL}}
\newcommand{\fl}{\mathrm{fl}}
\newcommand{\op}{\odot}
\newcommand{\Eabs}{E_{\mathrm{abs}}}
\newcommand{\Erel}{E_{\mathrm{rel}}}

\begin{document}

\pagestyle{fancy}
\fancyhead{}
\lhead{陈力豪, 3220103614}
\chead{数据结构与算法第五次作业}
\rhead{Oct.30th, 2024}

\section{$remove$函数设计思路}
经过修改的$remove$函数如下:
\begin{lstlisting}[language=c++]
    BinaryNode *detachMin(BinaryNode *&t){
        if(t == nullptr){
            return nullptr;
        }
        if(t->left == nullptr){
            BinaryNode *minnode = t;
            t = t->right;   //把当前节点改为右子树,在递归回归时可以自动接上
            return minnode;
        }
        return detachMin(t->left);
    }
    void remove(const Comparable &x, BinaryNode *&t) {
        if (t == nullptr) {
            return;  
        }
        if (x < t->element) {
            remove(x, t->left);
        } 
        else if (x > t->element) {
            remove(x, t->right);
        } 
        else {  
            if (t->left != nullptr && t->right != nullptr) {  
                BinaryNode *minNode = detachMin(t->right);  //找到右子树最小节点并移除,然后替代
                minNode->left = t->left;  
                minNode->right = t->right;  
                delete t;  
                t = minNode;  
            } 
            else {  
                BinaryNode *oldNode = t;
                t = (t->left != nullptr) ? t->left : t->right;  
                delete oldNode;
            }
        }
    }
\end{lstlisting}\par
首先设计了$detachMin$函数,找到子树中的最小节点,用一个$minnode$保存最小节点并返回,并把当前的$t$节点设置为$t$的右子树,这样在递归返回时不会破坏树的结构,能够自动接上这个右子树;然后将$remove$函数的逻辑改为先用$detachMin$函数找到右子树的最小节点并在右子树上移除这个节点,然后用这个最小节点替代原始节点,最后删除原节点,完成移除操作。
\section{测试函数设计和测试结果}
测试函数如下:
\begin{lstlisting}[language=c++]
    #include <iostream>
    #include "BinarySearchTree.h"
    
    int main() {
        BinarySearchTree<int> bst;
    
        // Inserting elements
        bst.insert(20);
        bst.insert(10);
        bst.insert(30);
        bst.insert(5);
        bst.insert(15);
        bst.insert(25);
        bst.insert(35);
    
        std::cout << "Initial tree (in-order traversal):" << std::endl;
        bst.printTree();
    
        std::cout << "\nRemoving a leaf node (5):" << std::endl;
        bst.remove(5);
        bst.printTree();
    
        std::cout << "\nRemoving a node with one child (10):" << std::endl;
        bst.remove(10);
        bst.printTree();
    
        std::cout << "\nRemoving a node with two children (30):" << std::endl;
        bst.remove(30);
        bst.printTree();
    
        std::cout << "\nRemoving the root node (20):" << std::endl;
        bst.remove(20);
        bst.printTree();
    
        std::cout << "\nAttempting to remove an element not in the tree (40):" << std::endl;
        bst.remove(40);
        bst.printTree();
    
        std::cout << "\nRemoving all elements to make the tree empty:" << std::endl;
        bst.remove(15);
        bst.remove(25);
        bst.remove(35);
        bst.printTree();
    
        return 0;
    }
\end{lstlisting}
测试思路为:先建立一棵树:
\begin{lstlisting}
        20
       /  \
      10  30
     / \  / \
    5  15 25 35
\end{lstlisting}
首先,删除叶子节点5:
\begin{lstlisting}
       20
      /  \
    10   30
     \   / \
     15 25 35
\end{lstlisting}
然后,删除有一个子节点的节点10:
\begin{lstlisting}
      20
     /  \
    15  30
       /  \
      25  35
\end{lstlisting}
然后,删除有两个子节点的节点30:
\begin{lstlisting}
      20
     /  \
    15  35
       /
     25
\end{lstlisting}
删除节点20:
\begin{lstlisting}
      25
     /  \
    15  35
\end{lstlisting}
测试删除一个不存在的节点,最后清空树。\par
测试结果如下,一切正常。
\begin{lstlisting}
    Initial tree (in-order traversal):
    5
    10
    15
    20
    25
    30
    35
    
    Removing a leaf node (5):
    10
    15
    20
    25
    30
    35
    
    Removing a node with one child (10):
    15
    20
    25
    30
    35
    
    Removing a node with two children (30):
    15
    20
    25
    35
    
    Removing the root node (20):
    15
    25
    35
    
    Attempting to remove an element not in the tree (40):
    15
    25
    35
    
    Removing all elements to make the tree empty:
    Empty tree
\end{lstlisting}
我用 valgrind 进行测试,发现没有发生内存泄露。
\begin{lstlisting}
    ==4852== 
    ==4852== HEAP SUMMARY:
    ==4852==     in use at exit: 0 bytes in 0 blocks
    ==4852==   total heap usage: 9 allocs, 9 frees, 74,920 bytes allocated
    ==4852== 
    ==4852== All heap blocks were freed -- no leaks are possible
    ==4852== 
    ==4852== For lists of detected and suppressed errors, rerun with: -s
    ==4852== ERROR SUMMARY: 0 errors from 0 contexts (suppressed: 0 from 0)      
\end{lstlisting}
\end{document}

%%% Local Variables: 
%%% mode: latex
%%% TeX-master: t
%%% End: 
