\documentclass[UTF8]{ctexart}
\usepackage{geometry, CJKutf8}
\geometry{margin=1.5cm, vmargin={0pt,1cm}}
\setlength{\topmargin}{-1cm}
\setlength{\paperheight}{29.7cm}
\setlength{\textheight}{25.3cm}

% useful packages.
\usepackage{amsfonts}
\usepackage{amsmath}
\usepackage{amssymb}
\usepackage{amsthm}
\usepackage{enumerate}
\usepackage{graphicx}
\usepackage{multicol}
\usepackage{fancyhdr}
\usepackage{layout}
\usepackage{listings}
\usepackage{float, caption}

\lstset{
    basicstyle=\ttfamily, basewidth=0.5em
}

% some common command
\newcommand{\dif}{\mathrm{d}}
\newcommand{\avg}[1]{\left\langle #1 \right\rangle}
\newcommand{\difFrac}[2]{\frac{\dif #1}{\dif #2}}
\newcommand{\pdfFrac}[2]{\frac{\partial #1}{\partial #2}}
\newcommand{\OFL}{\mathrm{OFL}}
\newcommand{\UFL}{\mathrm{UFL}}
\newcommand{\fl}{\mathrm{fl}}
\newcommand{\op}{\odot}
\newcommand{\Eabs}{E_{\mathrm{abs}}}
\newcommand{\Erel}{E_{\mathrm{rel}}}

\begin{document}

\pagestyle{fancy}
\fancyhead{}
\lhead{陈力豪, 3220103614}
\chead{数据结构与算法第四次作业}
\rhead{Oct.16th, 2024}

\section{测试程序的设计思路}

1.我首先创建了一个链表lst,使用$push\_back$插入了0到9的数字。\par
2.然后,使用$pop\_back$删除了链表的最后一个元素。\par
3.接着,使用$push\_front$插入了-1,和-1-1。\par
4.然后,使用$pop\_front$删除了链表的第一个元素,即-2。\par
5.然后,测试$clear$和$empty$函数,如果都正常,应该返回$True$。\par
6.然后,测试$insert$和$erase$函数,插入和删除元素。\\
这一步中,我首先创建了一个含两个元素5,10的链表,创建指向5的迭代器it,令++it(这里已经测试了前++),然后在这个位置$insert$了一个元素7和3+5,正常应返回5,7,8,10,然后--it(这里已经测试了前--),再调用$erase$,删除7,正常应返回5,7,10。\par
7.接下来,以上面这个5,10的链表作为被复制的链表,新建链表分别测试拷贝构造函数、赋值运算符、移动构造函数。\par
8.最后,新建一个含1,2元素的链表lst1,测试iterator的运算符重载,包括$==, !=, ++, --$。\par
\section{测试的结果}

测试结果如下,一切正常。
\begin{lstlisting}
    Testing push_back (push 0-9):
    0 1 2 3 4 5 6 7 8 9
    Testing pop_back (pop the last one):
    0 1 2 3 4 5 6 7 8
    Testing push_front (with item -1):
    -2 -1 0 1 2 3 4 5 6 7 8
    Testing pop_front (pop the first one):
    -1 0 1 2 3 4 5 6 7 8
    Testing clear and empty:
    List empty? true
    Testing insert and erase:
    create a List with 5 10:
    5 10
    Insert 7 after 5:
    5 7 8 10
    Erase 3+5:
    5 7 10
    Testing copy constructor:
    5 7 10
    Testing assignment operator:
    5 7 10
    Testing move constructor:
    5 7 10
    Testing move assignment operator:
    5 7 10
    Testing equality and inequality operators for iterator:
    1 2
    itr1 == itr2? true
    itr1 == itr3? false
    Testing ++ and -- operators:
    itr1++ == itr4? false
    itr1 == itr4? true
    itr1-- == itr2? false
    itr1 == itr2? true
\end{lstlisting}
我用 valgrind 进行测试,发现没有发生内存泄露。
\begin{lstlisting}
    ==3598== HEAP SUMMARY:
    ==3598==     in use at exit: 0 bytes in 0 blocks
    ==3598==   total heap usage: 38 allocs, 38 frees, 75,616 bytes allocated
    ==3598== 
    ==3598== All heap blocks were freed -- no leaks are possible
    ==3598== 
    ==3598== For lists of detected and suppressed errors, rerun with: -s
    ==3598== ERROR SUMMARY: 0 errors from 0 contexts (suppressed: 0 from 0)    
\end{lstlisting}
\end{document}

%%% Local Variables: 
%%% mode: latex
%%% TeX-master: t
%%% End: 
