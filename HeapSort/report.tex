\documentclass[UTF8]{ctexart}
\usepackage{geometry, CJKutf8}
\geometry{margin=1.5cm, vmargin={0pt,1cm}}
\setlength{\topmargin}{-1cm}
\setlength{\paperheight}{29.7cm}
\setlength{\textheight}{25.3cm}

% useful packages.
\usepackage{amsfonts}
\usepackage{amsmath}
\usepackage{amssymb}
\usepackage{amsthm}
\usepackage{enumerate}
\usepackage{graphicx}
\usepackage{multicol}
\usepackage{fancyhdr}
\usepackage{layout}
\usepackage{listings}
\usepackage{float, caption}
\usepackage{listings}
\usepackage[usenames,dvipsnames]{xcolor}
\lstset{
    basicstyle=\ttfamily, basewidth=0.5em
}
\definecolor{mygreen}{rgb}{0,0.6,0}
\definecolor{mygray}{rgb}{0.5,0.5,0.5}
\definecolor{mymauve}{rgb}{0.58,0,0.82}
\lstset{ %
backgroundcolor=\color{white},   % choose the background color
basicstyle=\footnotesize\ttfamily,        % size of fonts used for the code
columns=fullflexible,
breaklines=true,                 % automatic line breaking only at whitespace
captionpos=b,                    % sets the caption-position to bottom
tabsize=4,
commentstyle=\color{mygreen},    % comment style
escapeinside={\%*}{*)},          % if you want to add LaTeX within your code
keywordstyle=\color{blue},       % keyword style
stringstyle=\color{mymauve}\ttfamily,     % string literal style
frame=single,
rulesepcolor=\color{red!20!green!20!blue!20},
% identifierstyle=\color{red},
language=c++,
}

% some common command
\newcommand{\dif}{\mathrm{d}}
\newcommand{\avg}[1]{\left\langle #1 \right\rangle}
\newcommand{\difFrac}[2]{\frac{\dif #1}{\dif #2}}
\newcommand{\pdfFrac}[2]{\frac{\partial #1}{\partial #2}}
\newcommand{\OFL}{\mathrm{OFL}}
\newcommand{\UFL}{\mathrm{UFL}}
\newcommand{\fl}{\mathrm{fl}}
\newcommand{\op}{\odot}
\newcommand{\Eabs}{E_{\mathrm{abs}}}
\newcommand{\Erel}{E_{\mathrm{rel}}}

\begin{document}

\pagestyle{fancy}
\fancyhead{}
\lhead{陈力豪, 3220103614}
\chead{数据结构与算法第七次作业}
\rhead{Dec.1st, 2024}

\section{总体设计思路}
$heapSort$函数:先使用$make\_heap$对传入的$vector$操作,得到最大堆,然后使用$pop\_heap$逐次把最大的元素放到最后,完成堆排序。\par
$check$函数:遍历$vector$,判断是否满足从小到大排列。\par
测试过程:\par
1)生成随机序列;\par
2)使用STL自带的排序函数分别得到随机序列、升序序列、降序序列和部分元素重复序列。\par
3)分别使用我的堆排序算法和$std::sort\_heap$测试不同序列,计算时间并输出结果。\par
\section{测试结果}
程序运行测试结果如下:
\begin{table}[H]
    \centering
    \begin{tabular}{|c|c|c|}
    \hline
    \textbf{}           & \textbf{my heapsort time} & \textbf{std::sort\_heap time} \\ \hline
    random sequence     & 0.592942 s                & 0.41958 s                     \\ \hline
    ordered sequence    & 0.468792 s                & 0.419942 s                    \\ \hline
    reverse sequence    & 0.503647 s                & 0.422954 s                    \\ \hline
    repetitive sequence & 0.591673 s                & 0.419443 s                    \\ \hline
    \end{tabular}
\end{table}
\section{分析时间复杂度}
在我的代码中:\par
$make\_heap$函数将一个无序序列转换成堆。其时间复杂度是 O(n)。\par
$pop\_heap$函数将堆顶元素移到序列末尾,并调整堆。其时间复杂度是 O(log n)。\par
由于$pop\_heap$被调用了n-1次,所以总的时间复杂度是 O(nlog n)。\par

$std::sort\_heap$的时间复杂度也是是 O(nlog n)。\par
根据我查到的资料,$std::sort\_heap$可能会在执行时尽量避免多次不必要的堆结构调整,进一步优化了数据的交换和堆的重建过程;可能通过更高效的内存访问模式(例如缓存局部性优化)来加速操作;可能在元素交换上做了优化,避免了一些不必要的拷贝操作等,而我的算法则是完全按照堆排序的思路进行,所有的步骤都没有优化,使得在相同的时间复杂度下效果表现逊色于$std::sort\_heap$。
\end{document}

%%% Local Variables: 
%%% mode: latex
%%% TeX-master: t
%%% End: 
